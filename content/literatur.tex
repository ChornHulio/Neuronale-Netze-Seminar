\begin{thebibliography}{---}

% Einleitung
\bibitem[Kri07]{bib:dkriesel}
  \textsc{David Kriesel}: 
  \textbf{Ein kleiner Überblick über Neuronale Netze}.
  \emph{www.dkriesel.com}, 2007, Abgerufen am 15.04.2011

% Grundlagen

\bibitem[enc11]{bib:neuron}
  \textsc{encefalus.com}: 
  \textbf{Modell eines neuronalen Netzes}.
  \emph{www.encefalus.com}, 2010, Abgerufen am 17.04.2011
  
 \bibitem[hef11]{bib:stufe}
  \textsc{Klaus Heft}: 
  \textbf{MATHEMATISCHER VORKURS zum Studium der Physik}.
  \emph{www.thphys.uni-heidelberg.de/\textasciitilde{}hefft/vk1/}, 2011, Abgerufen am 17.04.2011

% Dreher
\bibitem[ct211]{bib:ct211}
  \textsc{Hans-Arthur Marsiske}: 
  \textbf{Wie Maschinen denken lernen}.
  c't Magazin, Ausgabe 2/2011, S. 70 ff., Heise Zeitschriften Verlag

\bibitem[ct711]{bib:ct711}
  \textsc{Hans-Arthur Marsiske}: 
  \textbf{Der WM-Titel ist näher gerückt}.
  c't Magazin, Ausgabe 7/2011, S. 72 ff., Heise Zeitschriften Verlag

\bibitem[Ame11]{bib:ameisen}
  \textsc{Karl-Werner Hansmann, Nils Boysen}: 
  \textbf{Ameisenalgorithmus}.
  \emph{www.ameisenalgorithmus.de}, 2004, Abgerufen am 16.04.2011

% Henn

\end{thebibliography}

