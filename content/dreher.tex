\section{Erkenntnisse von Tobias Dreher}
Der Versuch das menschliche Gehirn im Computer nachzubilden, stellt die Informatik vor enorme Herausforderungen. Nach über 60 Jahren Erforschung und Entwicklung, können die Ergebnisse in Teilbereichen sinnvoll und effizient eingesetzt werden. Jedoch kann eine künstliche Intelligenz noch längst keinen Menschen ersetzen. Ob man das jemals schafft und ob man das überhaupt will, wird allerdings kontrovers diskutiert. Michael Beetz, der das Exzellenzcluster \emph{Cognition for Technical Systems} an der TU München mit koordiniert, antwortet auf die Frage, ob Roboter denken oder Denken nur simulieren mit: ``Wenn ich unter Denken ein Wissen verstehe, das ich nicht explizit habe, sondern aus anderem Wissen herleite, in eine explizite Form bringe und in konkrete Aktionen umsetze, dann würde ich sagen, Roboter denken schon heute. Wenn ich Denken als ganzheitlichen Prozess betrachte, wie er bei Menschen stattfindet, dann würde ich hoffen, dass Roboter nie denken.'' \cite[S. 75]{bib:ct211} Allerdings können weitere Errungenschaften auf dem Gebiet der künstlichen Intelligenz Menschenleben retten oder zumindest Menschen helfen.

Hier gibt es zwei Beispiele beim \emph{RoboCup}, die Weltmeisterschaft der Roboter im Fußball. Neben den Fußballturniere gibt es mehrere Wettbewerbe in denen Roboter gegeneinander antreten um spezielle Fähigkeiten zu testen. Eine Disziplin ist hierbei das Retten von Menschen nach Naturkatastrophen. Die Roboter in der sogenannten \emph{Rescue League} müssen sich in einem simulierten Katastrophengebiet autonom zurechtfinden, detailliertes Kartenmaterial erstellen und Rettungskräfte die Position von Überlebenden mitteilen. Bis solche autonom agierende Roboter im Ernstfall eingesetzt werden können, wird es noch Jahre dauern, jedoch kann ein schneller Fortschritt dem Menschen nur von Vorteil sein. Das zweite Beispiel stellt die Liga \emph{RoboCup@Home} dar. Hier müssen sich die Roboter in einer Wohnumgebung autonom zurechtfinden und mit Menschen kommunizieren können. Dies kann in Hinblick auf die alternde Gesellschaft Menschen den Alltag erleichtern. \cite[S. 73]{bib:ct711}

Neuronale Netze sind der Natur nachempfunden, da die Natur unglaublich effizient und fehlertolerant arbeitet. So ist es ein logischer Schritt noch mehr von der Natur zu kopieren. Als eines der bekanntesten Beispiele sei hier der Ameisenalgorithmus genannt. Dabei werden Ameisen simuliert, die den Anspruch haben, den kürzesten Weg zum Futter zu finden. Dieser zeigt anschaulich, wie viel Potential in einfachen Algorithmen der Natur steckt. Verwendet wird der Ameisenalgorithmus zum Beispiel bei Navi\-gations\-soft\-ware und bei Prozessoptimierung im Maschinenbau. \cite{bib:ameisen}

Persönlich bleibt festzuhalten: Sowohl die Informatik, als auch die Biologie sind fas\-zi\-nie\-rend.  Die Faszination lässt sich nur steigern durch die Kombinationen zwischen den auf den ersten Blick sehr unterschiedlichen Gebiete. Dieses Potential, das hier frei wird, gilt es weiter zu erforschen und zu nutzen.