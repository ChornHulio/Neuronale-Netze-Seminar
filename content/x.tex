\section{Einleitung}
In dieser Arbeit wird der Zusammenhang zwischen den Innovationen bei Smartphones und den Marktvolumen der einzelnen Modelle und Hersteller erläutert. Hierbei dient der Konzern \emph{Nokia} als Beispiel. Im einleitenden Kapitel wird die Motivation, sowie die einzelnen Ziele und Fragestellungen, die der Arbeit zugrunde liegen, dargestellt.

\subsection{Motivation}

Laut dem Branchenindex \emph{GfK TEMAX} sei inzwischen jedes sechste verkaufte Mobiltelefon ein Smartphone. Das Marktforschungsunternehmen \emph{Gartner} zeigt in einer aktuellen Studie auf, dass sogar nahezu jedes fünfte verkaufte Neugerät ein Smartphone sei. Obwohl der Mobilfunkmarkt im letzten Jahr eine Wachstumsrate von rund 35 \% aufweist und sich der Absatz bei Smartphones fast verdoppelt hat, hat der Weltmarkführer \emph{Nokia} und deren eingesetztes Smartphone-Betriebssystem \emph{Symbian} Probleme sich an der Spitze zu halten.

Bei dieser groben Darstellung der aktuellen Marktlage stellt sich die Frage, ob sich hier ein Zusammenhang finden lässt. Warum scheinen ehemalig sehr erfolgreiche Firmen den Anschluss zu verlieren? Was unternehmen diese Firmen dagegen? Und warum ist gerade \emph{Android} (\emph{Google}) und \emph{iOS} (\emph{Apple}) so erfolgreich?

\subsection{Ziele}

Ziel dieser Arbeit ist es ein Grund für die schwachen Absatzzahlen bei \emph{Nokia} zu finden. Hierbei soll auch die verwendete Hardware, sowie darauf aufsetzende Betriebssysteme genauer betrachtet und auf deren Innovationskraft hin analysiert werden. Anhaltspunkte sind hier
\begin{itemize}
	\item Hardware-Perfomance
	\item Display
	\item Ausstattung
	\item Bedienbarkeit
\end{itemize}
