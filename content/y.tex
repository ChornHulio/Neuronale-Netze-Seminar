\section{Grundlagen}

In diesem Kapitel werden Begriffe und Grundlagen erläutert, die in dieser Arbeit verwendet werden. Hierzu zählt sowohl die terminologische Abgrenzung des Begriffs \emph{Smartphone}, als auch die Techniken hinter den Betriebssystemen \emph{Symbian} und \emph{MeeGo}.

\subsection{Smartphones}

Smartphones unterscheiden sich von herkömmlichen Handys insbesondere durch folgende Fä\-hig\-kei\-ten:
\begin{itemize}
	\item bessere Kalenderfunktionen
	\item Internetbrowser
	\item Erweiterbarkeit durch zusätzliche Programme
	\item leistungsfähigere Hardware
	\item größere Displays
	\item evtl. Touchscreens
\end{itemize}

\subsection{Qt}
\emph{Qt} ist ein Framework (Klassenbibliothek) für die Entwicklung von grafischen Oberflächen, die auf verschiedenen Hardwarearchitekturen und Betriebssystemen einsetzbar sind. Außer der komfortablen Erstellung von Oberflächen, bietet \emph{Qt} auch zusätzliche Funktionalitäten an. Hierzu zählen unter anderem Netzwerkzugriff, XML-Unterstützung, Datenbankanbindung und Internationalisierung.

Ursprünglich entwickelt wurde \emph{Qt} von \emph{Trolltech}. Die Firma wurde im Jahr 2008 von \emph{Nokia} aufgekauft. \emph{Nokia} verfolgt das Ziel \emph{Qt}-Applikationen auf ihren neuen Smartphones zu unterstützen. Momentan laufen die mittels \emph{Qt} entwickelten Programme unter folgenden Smartphone-Betriebssystemen:
\begin{itemize}
	\item Symbian S60
	\item Symbian 3
	\item Maemo 5
	\item MeeGo
\end{itemize}
Seit der Version 4.6 unterstützt \emph{Qt} auch Multitouch und Gestensteuerung. Im April 2010 stellte \emph{Nokia} eine Erweiterung von \emph{Qt} namens \emph{Qt Mobility} vor. \emph{Qt Mobility} ist eine weitere Ansammlung von Bibliotheken, die speziell für Mobilgeräte entwickelt wurden, aber auch im Desktopbereich eingesetzt werden können. Hierzu zählen unter anderem Klassen für den Nachrichtenaustausch, Multimedia, Kontaktverwaltung, Ortsbestimmung, und spezielle Sensoren.

\subsection{MeeGo}
Das Ende Oktober 2010 in Version 1.1 erschienene Betriebssystem \emph{MeeGo} entstand durch den Zusammenschluss zweier Linux-Dis\-tri\-bu\-tionen. \emph{Intel} steuerte dem Projekt das Net\-book-Betriebs\-system \emph{Moblin} bei. \emph{Nokia} brachte ihr Smartphone-Betriebssystem \emph{Maemo 5} ein. So ist \emph{MeeGo} für Smartphones, Netbooks, Tablets, Fernseher und Infotainmentsysteme im Auto gedacht. Es besteht aus einem Kern und einem darauf aufbauendem API. Sogenannte UX setzen wiederum auf dem \emph{MeeGo API} auf.

Da in \emph{MeeGo} ein aktueller Linux Kernel verwendet wird, läuft das Betriebssystem sowohl auf ARM-, als auch auf x86-Architekturen. Daher rührt \emph{Intel}'s Interesse an \emph{MeeGo}. \emph{Nokia} ist bemüht ein Großteil der Funktionalität mittels \emph{Qt} und \emph{Qt Mobility} zu realisieren. Die {Smartphone-UX beinhaltet jedoch noch einige Fehler und wird als Testversion angesehen. Im April 2011 wird mit \emph{MeeGo 1.2} die erste finale Version erwartet.
