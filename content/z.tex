\section{Marktentwicklung}

In den vergangen Jahren hat \emph{Nokia} stark Marktanteile bei Smartphones verloren. Die Abbildung zeigt, dass \emph{Nokia}'s Marktanteil von 48,7\% im dritten Quartal 2007 auf 39,3\% im dritten Quartal 2009 gefallen ist. Zu dem Jahr 2010 sind keine Zahlen verfügbar, jedoch kann aus Abbildung einen weiteren Verlust gefolgert werden. Bei den Herstellern haben insbesondere \emph{Apple} mit ihrem \emph{iPhone} und \emph{Research In Motion (RIM)} mit der \emph{Blackberry}-Serie deutliche Zuwächse verzeichnen können.

In der Abbildung \ref{fig:osStatistik} sind die Marktanteile der Smartphone-Betriebssysteme vom dritten Quartal 2008 bis heute dargestellt. Das unter anderem von \emph{Nokia} eingesetzte \emph{Symbian} fällt von 49,8\% auf 36,6\%. Hier nimmt besonders \emph{Android}, aber auch \emph{iOS}, den Konkurrenten Marktanteile ab. Ein Teil des Verlustes von \emph{Symbian} im Jahr 2010 ist auf den Ausstieg von \emph{Samsung} und \emph{Sony-Ericsson} zurückzuführen. Diese zwei Hersteller verkaufen momentan nur noch wenige \emph{Symbian}-Modelle.
